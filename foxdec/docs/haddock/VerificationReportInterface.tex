\haddockmoduleheading{VerificationReportInterface}
\label{module:VerificationReportInterface}
\haddockbeginheader
{\haddockverb\begin{verbatim}
module VerificationReportInterface (
    Retrieve, FunctionEntry, InstructionAddress, ctxt_read_report,
    retrieve_io, ctxt_get_function_entries,
    ctxt_get_instruction_addresses, ctxt_get_indirections,
    ctxt_get_instruction, ctxt_get_invariant,
    ctxt_get_internal_function_calls, ctxt_get_cfg
  ) where\end{verbatim}}
\haddockendheader

The interface to the \haddocktt{.report} generated after running FoxDec.
After running FoxDec, a "verification report" (an object of type \haddocktt{\haddocktt{Context}}) can be retrieved from the generated .report file (see function \haddocktt{ctxt{\char '137}read{\char '137}report}).
Essentially, this module provides hooks into some of the information retrieved and derived from the binary,
including instructions, invariants, function entry points, etc.\par
A verification report is represented by the type \haddocktt{\haddocktt{Context}}, as internally it
is just the context passed around and maintained during verification.\par
The main flow is to read the .report file and use these functions to retrieve information.
The following example reads in a .report file provided as first command-line parameter and outputs the function entries:\par
\begin{quote}
{\haddockverb\begin{verbatim}
 main = do
   args <- getArgs
   ctxt <- ctxt_read_report $ head args
   putStrLn $ show $ ctxt_get_function_entries ctxt\end{verbatim}}
\end{quote}
Some of the information is automatically also exported in plain-text format, for easy access.\par
\begin{haddockdesc}
\item[\begin{tabular}{@{}l}
type Retrieve a = Context -> Either String a
\end{tabular}]
{\haddockbegindoc
The return type when retrieving information from a verification report: either an error message or a result.\par}
\end{haddockdesc}
\begin{haddockdesc}
\item[\begin{tabular}{@{}l}
type FunctionEntry = Int
\end{tabular}]
{\haddockbegindoc
Function Entries are simply integers\par}
\end{haddockdesc}
\begin{haddockdesc}
\item[\begin{tabular}{@{}l}
type InstructionAddress = Int
\end{tabular}]
{\haddockbegindoc
Instruction Addresses are simply integers\par}
\end{haddockdesc}
\begin{haddockdesc}
\item[\begin{tabular}{@{}l}
ctxt{\char '137}read{\char '137}report
\end{tabular}]
{\haddockbegindoc
\haddockbeginargs
\haddockdecltt{::} & \haddockdecltt{String} & The filename \\
\haddockdecltt{->} & \haddockdecltt{IO Context} & \\
\end{tabulary}\par
Read in the .report file from a file with the given file name.
   May produce an error if no report can be read from the file.
   Returns the verification report stored in the .report file.\par}
\end{haddockdesc}
\begin{haddockdesc}
\item[\begin{tabular}{@{}l}
retrieve{\char '137}io :: Either String a -> IO a
\end{tabular}]
{\haddockbegindoc
Retrieve information from a \haddocktt{\haddocktt{Context}} read from a .report file, or die with an error message.
 For example:\par
\begin{quote}
{\haddockverb\begin{verbatim}
do
  ctxt <- ctxt_read_report filename
  retrieve_io $ ctxt_get_instruction a ctxt\end{verbatim}}
\end{quote}
This code reads in a .report file with the given \haddocktt{filename},  and reads the instruction at address \haddocktt{a} if any.\par}
\end{haddockdesc}
\begin{haddockdesc}
\item[\begin{tabular}{@{}l}
ctxt{\char '137}get{\char '137}function{\char '137}entries :: Retrieve (Set FunctionEntry)
\end{tabular}]
{\haddockbegindoc
Retrieve all function entries.\par
Returns a set of funtion entries.\par}
\end{haddockdesc}
\begin{haddockdesc}
\item[\begin{tabular}{@{}l}
ctxt{\char '137}get{\char '137}instruction{\char '137}addresses :: Retrieve (Set InstructionAddress)
\end{tabular}]
{\haddockbegindoc
Retrieve all instruction addresses.\par
Returns a set of instruction addresses.\par}
\end{haddockdesc}
\begin{haddockdesc}
\item[\begin{tabular}{@{}l}
ctxt{\char '137}get{\char '137}indirections :: Retrieve Indirections
\end{tabular}]
{\haddockbegindoc
Retrieve all indirections\par
Returns a mapping that provides for some instruction addresses a set of jump targets.\par}
\end{haddockdesc}
\begin{haddockdesc}
\item[\begin{tabular}{@{}l}
ctxt{\char '137}get{\char '137}instruction :: InstructionAddress -> Retrieve (Instr, String)
\end{tabular}]
{\haddockbegindoc
Retrieve instruction for a given instruction address, both as datastructure and pretty-printed\par}
\end{haddockdesc}
\begin{haddockdesc}
\item[\begin{tabular}{@{}l}
ctxt{\char '137}get{\char '137}invariant :: FunctionEntry -> InstructionAddress -> Retrieve Pred
\end{tabular}]
{\haddockbegindoc
Retrieve invariant for a given function entry and instruction address\par
An invariant is a predicate provding information over registers, memory, flags, and verification conditions.\par}
\end{haddockdesc}
\begin{haddockdesc}
\item[\begin{tabular}{@{}l}
ctxt{\char '137}get{\char '137}internal{\char '137}function{\char '137}calls :: FunctionEntry -> Retrieve (Set FunctionEntry)
\end{tabular}]
{\haddockbegindoc
Retrieve all internal function calls for a given function entry\par
Returns a set of function entries.\par}
\end{haddockdesc}
\begin{haddockdesc}
\item[\begin{tabular}{@{}l}
ctxt{\char '137}get{\char '137}cfg :: FunctionEntry -> Retrieve CFG
\end{tabular}]
{\haddockbegindoc
Retrieve a CFG for a given function entry\par}
\end{haddockdesc}